\documentclass{letter}
\usepackage{graphicx}


\newcommand{\Otitle}{\emph{OpenCFU, a New Free and Open-Source Software to Count
Cell Colonies and Other Circular Objects}}


\newcommand{\CMinorBalancedChanges}{c2 and c4 @page2}
\newcommand{\CMajorBalancedChanges}{c6 @p2}
\newcommand{\CVersatDescrib}{c1 @p3; c3 and c4 @p5; c1 and c2 @p7 and figure 6}
\newcommand{\COutputDescrib}{c2 @p4}
\newcommand{\CDTDescrib}{c5 @p8}
\newcommand{\CCircuForm}{c1 and c3 @p8}



\interfootnotelinepenalty=10000 \longindentation=0pt

\signature{ Quentin Geissmann\\
PhD. student\\
Institute for Biology\\
Free University of Berlin\\
}

\address{ Institute for Biology\\
Free University of Berlin\\
K\"{o}nigin-Luise-Strasse 1-3\\
D-14195 Berlin, Germany\\
q.geissmann@fu-berlin.de}
\name{Quentin Geissmann}


\begin{document}

\begin{letter}{}
\opening{Dear editors,}
Thank you for having reconsidered my manuscript entitled ``\Otitle{}'' for publication in PLoS ONE.
My first revisions were extensive and 
involved improvements to 
both the program and the manuscript.
As far as I understand, both anonymous reviewers were satisfied by these modifications.
However, they respectively requested that I provide additional information
to the reader and that I answer some questions
related to methodological issues. More importantly, the academic editor has pointed out the
inadequacy of some of my qualitative comparisons to other methods. 
In this second resubmission, I provide a revision to my manuscript addressing all these issues.

In my first submission, I only compared my method against one alternative and omitted to
mentions some tools, nor did I explain why I could not compare some of them to my method.
In response to 
these 
criticisms--which I think were fair--,
I endeavoured to try additional methods and explain why I could not include some of them.
Some of the statements regarding other methods present in my first resubmission were judged inappropriate.
I was therefore requested to provide a more scientifically balanced discussion over other methods.

For example:
\begin{quote}
The academic editor wrote:\\
``[\ldots]with respect to your writing `Clono-counter [12] could not be used because of critical bugs',
I tried to download the tool myself: it ran flawlessly on the provided example data.
If the key issue here is that you need to reduce the resolution of your
input data before loading them into Clono-counter,
stating that the program has `critical bugs' seems inappropriate.''\\
\end{quote}

I admit that the use of the 
phrase
``critical bug'' to define
some of the flaws of Clono-counter
was perhaps disproportionate and inaccurate. 
In my second resubmission, I have rephrased this statement into a more balanced one.
I maintain
that there are some fundamental issues with this method 
(including bugs) that are critical for adoption and performance. 
However, I do not think this letter (or my manuscript) are the place to expand on this subject. I will, nevertheless, be happy to elaborate a detailed report about this method if you request it.
 
Another example was my statement about programs that appeared unavailable.
\begin{quote}
The academic editor wrote:\\
``[\ldots]it seems reasonable to believe that these tools have been useful for some authors, so it would be fair to discuss these tools in a scientifically balanced manner.''
\end{quote}
I have also rephrased my statement about CHiTA and Arraycount. 
I would genuinely like to agree with the academic editor about 
the fact that they might have been useful to some authors.
However, I have been reading studies citing respectively ChiTA\footnote{There are 13 according to google scholar.}
and Arraycount\footnote{There are five according to google scholar.} and could not find any
author who mentioned, explicitly or implicitly, that they have used these programs. 
In fact, most often, they cite CHiTA and Arraycount only once in the text 
in order to give an example of colony/cell counting program. 
I therefore have no evidence that these tools were actually used by other authors.
I understand the necessity of being balanced in a scientific paper.
However, I have to mention the existence of these tools, but if I do so, 
I must explain why I did not include them in my comparison.
The reason is 
that 
they are, as far as I am aware, unavailable. 
I also feel that I have been objective when I described that these programs
do not have a download link,
are not released in a public repository or enclosed as supplementary materials.
In addition, considering the large amount of time I 
have spent
trying to download them, 
it would seem useful to inform the readers about 
their 
unavailability.


I was also asked to mention the related work by Brugger et al. (2012).
\begin{quote}
The academic editor wrote:\\
``Also, it would be appropriate to discuss the closely related work by Brugger et al. (2012)
 - the installation problems do not justify ignoring their work and methodology altogether, 
 and it seems not appropriate to discuss installation problems in a scientific paper (I checked myself, and it doesn't work for me either,
  the software requires an old v7.3 of the MCR - 
 perhaps you can ask the authors to recompile or provide you with the right .dll?)''
\end{quote}
 I cite this work in my present resubmission.
 My understanding was that the program will hardly work without the specific 
 capture device built by the authors.
 I have contacted the corresponding author 
who  
confirmed this.
 He however also indicated 
to
me that the program can be used to analyse digital
images thanks to a script\footnote{Present the file ``cc\_test.bat''.}.
 After some efforts, I was able to install the program.
 However, I was not able 
to analyse my own raw samples.
 It turns out that the program will only work 1441x1441px ``tif'' images with 
a dark background.  More importantly, the script does not give
the option to use a mask or to modify any parameters. This leads, in my case, to detections of the edge of the dish as colonies. I will be happy to provide a detailed discussion about the usability of the script for
 biologists if required.

In addition to having rephrased statements described as scientifically unbalanced, I have tried to briefly
introduce each method by describing the originality of its image processing. This should bring
another perspective to the reader and
make my descriptions more 
balanced\footnote{The changes related to my statements about
other methods are marked-up as \CMinorBalancedChanges{} and, 
more importantly, \CMajorBalancedChanges{}.}.

I was asked to describe the result outputs in my manuscript:
\begin{quote}
Reviewer 1 wrote:\\
`` I believe that the author should at least describe the output of an analysis.
 It is clear that one can retrieve the number of colonies and a more detailed output
  including information about individual colonies, but what does this include?''
\end{quote}
In my present resubmission, 
I include a new paragraph describing the output\footnote{See change  \COutputDescrib{}.}.
 
I was also asked to inform the reader about other possible applications
of OpenCFU.
\begin{quote}
Reviewer 1 wrote:\\
``  From the manuscript the reader gets a good idea of the speed and
 reliability of the tool but is not well 
informed on the output and possible applications beyond the counting of colonies.''
\end{quote}
To answer this criticism, I included 
new figure (fig. 6) demonstrating the capacity of OpenCFU to 
enumerate objects such as round seeds and pollen. This figure is
non exhaustive and 
merely qualitative. For this reason, I would be happy to 
include it as a supplementary material instead. 
This figure substantiates my claim that OpenCFU can be used to enumerate objects other than cell colonies. 
I have also consequently changed and added some statements to my manuscript\footnote{See changes \CVersatDescrib{}.}.

I was asked a technical question about the distance transform:
\begin{quote}
Reviewer 2 wrote:\\
`` a Eucledian DT with a mask? this is somewhat strange to me; 
normally this is parameter free; no kernel mask is given as the DT is applied over the image/subimage that is asked for.''
\end{quote}
 
 To compute the distance transform, I used the OpenCV build in 
 function\footnote{\emph{i.e.} \texttt{cv::distanceTransform()} with \texttt{CV\_DIST\_L2} and a maskSize of 5.}.
 According to OpenCV documentation: 
 ``For the CV\_DIST\_C and CV\_DIST\_L1 types, the distance is calculated precisely, whereas for CV\_DIST\_L2 (Euclidean distance) the distance can be calculated
  only with a relative error (a  $5\times5$ mask gives more accurate results).''
  As an approximation of Euclidean distance transform, OpenCV implements the algorithm
  describe by Borgesfors (1986) with a=1, b=1.4, c=2.1969. 
 I understand that describing this as an 
``Euclidean distance transform'' is ambiguous and inaccurate.
 My revised submission explains this more precisely\footnote{See changes \CDTDescrib{}.}. 
 As far as I have experienced, using a true Euclidean distance
 transform did not improve distinctively the accuracy 
of the algorithm.
  
%~~~~~~~~~~~~~~~~~~~~~~~~~~~~~~~~~~~~~~~~~~~~~~~~~~~~~~~~~~~~~~~~~~~~~~~~~~~~~~~~~~~~~~~~~~~~~~~~~~%
I was asked a technical question about 
how the values of threshold for the particle filter were obtained:
\begin{quote}
Reviewer 2 wrote:\\
`` why aspect ratio and not the Form Factor that compares the shape to a circle; a cicle is always 1
 this is accomplished by multiplying the Area with 4PI.''
\end{quote}
To be clear, I use both compactness and aspect ratio.
Because circularity is more common than compactness 
I transformed the formulae given in my material and 
method\footnote{See changes \CCircuForm{}.}.
This only affects readability because compactness is proportional to 
circularity\footnote{Here I use the same terminology as
Montero (2009) ``State of the Art of
Compactness and Circularity Measures''}.
Then, circularity is defined by the invert of form factor. 
Therefore, applying a threshold $t$ on circularity is
exactly similar to applying a threshold $1/t$ on form factor. 	

Under the assumption of an isolated colony, both aspect ratio and circularity are important. 
For instance, a cluster of two perfectly circular colonies of the same size only touching
by a single point will have a theoretical circularity of one (equal to a circle) whilst its
aspect ratio will be two. 
In this case, aspect ratio is an easy and safe way to reject some object as an isolated colony.

When the object is a cluster of colonies, the circularity can
considerably increase because colonies overlap. 
This means that a more tolerant threshold on circularity is needed for clustered colonies.
However, as the threshold on circularity is relaxed, 
it becomes possible to detect portions of edges as cluster of colonies.
In this case, using aspect ratio can be very useful because portions
 of edges have typically a very high aspect ratio whilst colonies normally do not cluster in a line.


%~~~~~~~~~~~~~~~~~~~~~~~~~~~~~~~~~~~~~~~~~~~~~~~~~~~~~~~~~~~~~~~~~~~~~~~~~~~~~~~~~~~~~~~~~~~~~~~~~~%
I was asked a technical question about 
the values of threshold for the particle filter:
\begin{quote}
Reviewer 2 wrote:\\
``  how are the ratio values chosen? ''\\
``  if circumstances change what happen these values. how are they obtained. ''
\end{quote}
The values were \emph{a priori} tested on a small library of binary images containing
manually drawn samples. This helped to define a range of plausible values for the parameters. 
Then, the particle filter was implemented in the main algorithm,
and the algorithm was tested on a range of different sample pictures 
(essentially the same as the sample pictures provided in OpenCFU's website). 
Importantly, the pictures
used to optimise parameter values 
are different from the set of pictures used to compare methods/humans.
The robustness to jpeg compression, down-scaling and different noises was also taken in consideration.
The values kept gave satisfactory results for all my samples.
I cannot exclude that these values will 
give poor results
in some unpredicted cases.
However, it would be very simple to recompile my program in order to adapt these thresholds.

%~~~~~~~~~~~~~~~~~~~~~~~~~~~~~~~~~~~~~~~~~~~~~~~~~~~~~~~~~~~~~~~~~~~~~~~~~~~~~~~~~~~~~~~~~~~~~~~~~~%
In conclusion, I have taken into serious consideration all the criticisms
and questions raised by the academic editor and the two anonymous reviewers.
I have endeavoured to rewrite more balanced statements about other methods.
I also provide 
more information
to the reader regarding the result outputs
of OpenCFU and its possible utility 
beyond colony counting.
Finally, I answered technical questions about the methods.
Altogether, the suggested revisions have, I believe, improved the quality of my
work and made my manuscript more suitable for publication in PLoS ONE.


\closing{Yours faithfully,
}
\ps{\emph{p.s.} The changes since the last revision 
are marked-up in the present submission.
Some very minor changes (punctuation and typos) have not been marked-up,
 but all the others have.
Also, the reference are corrupted (they appear as a non-readable
characters rather than a number in brackets) in the marked-up copy.
They will obviously be corrected in the next/final submission.
}

\end{letter}
\end{document}

	
	
	% 
	% I would however like to clarify what are the issues with this specific method:
	% 
	% It is very hard to deal with large image in Clono-counter.
	% As a workaround, the academic editor suggested to simply down-scale the images before analysis.
	% This solution will work if the colonies in the images are large and well defined,
	% but will result in a very significant loss of information relative to small objects.
	% In addition, many user will be interested by analysing many (sometimes hundreds) pictures.
	% However, in my experience, experimental biologists will rarely know how to efficiently rescale multiple images.
	% 
	% I also mentioned that opening multiple files seems impossible (even from command line).
	% To me, this is a very serious flaw if the user has many files to deal with. For each file,
	% the biologist will have to do, at least, six manual interventions: choose a file to open, draw a mask, 
	% set values for the three parameters and manually write down the result. 
	% These six steps are all very error-prone and time consuming. For instance, it is very easy
	% to click on the wrong file, or enter a wrong parameter value over hundred of pictures.
	% 
	% From a user perspective, this flaws undermine some of the benefits of the semi-automatic approach when dealing with multiple images.
	% From my point of view, I cannot include the method in the comparison.
	% I would compare methods that analysed different images (original \emph{vs.} down-sampled). 
	% I will be also unable to test speed because my test involves very large images without masks.
	% Finally, it is impossible to draw the same mask for consecutive images. Therefore, I cannot perform my ``robustness experiment''.
	% 
	% The program has other serious flaws/bugs that I did not mention. For instance, as you can see on my attached
	% ``cc1.tif'',  the results can be read as 
	%  $Total = Valid + ToSmall$. In this example, Clono-counter has found 12 colonies. This interpretation is intuitive and corresponds to the figure 1 of Niyazi \emph{et al.}'s paper(2007). 
	% However, when another mask is drawn, the number of valid colonies can be surprisingly negative (cf. my attached ``cc2.tif'').
	% 
	% This behaviour is not limited to my own picture, it happens with the example data. 
	% On one hand, I understand very well that my role as the author of the
	% present paper is not to criticise other methods in this manner or this extent.
	% On the other hand I was asked to explain how the GUI of
	% my program provides advantages over other tools. 
	% I feel that it is very hard to explain the advantages of my program
	% without mentioning flaws of other ones (especially serious ones).
	% I hope that my present statement is balanced enough.
	%~~~~~~~~~~~~~~~~~~~~~~~~~~~~~~~~~~~~~~~~~~~~~~~~~~~~~~~~~~~~~~~~~~~~~~~~~~~~~~~~~~~~~~~~~~~~~~~~~%
